%\chapter{Algumas Demonstra{\c c}\~oes}
%\chapter{Estimação da posição e orientação}
\chapter{Estimação da \textit{pose} em tempo real} 
 
\section{ViSP - Visual Servoing Platform}

Para a implementação do modo de controle por Servo Visão, foi utilizada a biblioteca ViSP. A ViSP contém um módulo de visão computacional que permite computar a \textit{pose} de um objeto a ser reconhecido por meio de um padrão pre-determindado. São utilizados algoritmos robustos e fornecidos mecanismos para calibração da câmera. Esse módulo é um envoltório para a biblioteca OpenCV com algoritmos focados em aplicações de robótica.

\begin{itemize}
\item Linear Lagrange approach (test is done to switch between planar and non planar algorithm)
\item Linear Dementhon approach (test is done to switch between planar and non planar algorithm)
\item Lowe aproach based on a Levenberg Marquartd non linear minimization scheme that needs an initialization from Lagrange or Dementhon aproach
\item  Non linear Lowe aproach initialized by Lagrange approach
\end{itemize}

\section{OpenCV}
