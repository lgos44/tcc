\chapter{Implementação}
A implementação dos algoritmos de controle detalhados nos capítulos anteriores for feita como extensão ao software RobotGUI desenvolvido pela equipe do LEAD-GSCAR, idealizado por Alex F. Neves Msc. 

\section{Motivação} 
Modular, genérico, ...

\section{Ferramentas}
Os seguintes softwares e \textit{frameworks} foram utilizados:

\begin{itemize}
\item Linux (Ubuntu) como Sistema Operacional
\item C++ como linguagem de programação
\item ROS como \textit{framework} principal utilizado o RobotGUI, fornecendo comunicação entre nós através de mensagens e serviços. Será descrito mais detalhadamente na próxima seção.
\item Qt como \textit{framework} para elaboração da interface gráfica. 
\end{itemize}

\section{ROS}

\section{Conceitos}

Primeiramente define-se como Computador Base aquele que será utilizado pelo operador para controlar e visualizar dados do robô. Define-se Computador Embarcado, ou do Robô aquele que está no robô conectado a todos os equipamentos, sensores e atuadores. O software executa módulos diferentes no robô e na base.

A arquitetura do RobotGUI baseia-se nos seguintes conceitos principais:

\begin{itemize}
\item Componentes: Lidam com a comunicação e processam dados no Computador Base.
\end{itemize}

\section{Doris Manipulator}

\section{Modos de Controle}
\begin{itemize}
\item Velocidade nas Juntas
\item Posição nas Juntas
\item Posição Espaço Operacional
\end{itemize}
