\chapter{Código} \label{chap:code}

Neste Apêndice são mostrados alguns trechos chave do código utilizado na implementação do controle no software. 

\definecolor{dkgreen}{rgb}{0,0.6,0}
\definecolor{dred}{rgb}{0.545,0,0}
\definecolor{dblue}{rgb}{0,0,0.545}
\definecolor{lgrey}{rgb}{0.95,0.95,0.95}
\definecolor{gray}{rgb}{0.2,0.2,0.2}
\definecolor{darkblue}{rgb}{0.0,0.0,0.6}
\lstdefinelanguage{cpp}{
      backgroundcolor=\color{lgrey},  
      basicstyle=\footnotesize \ttfamily \color{black} \bfseries,   
      breakatwhitespace=false,       
      breaklines=true,               
      %captionpos=b,                   
      commentstyle=\color{dkgreen},   
      deletekeywords={...},          
      escapeinside={\%*}{*)},                  
      frame=single,                  
      language=C++,                
      keywordstyle=\color{dblue},  
      morekeywords={BRIEFDescriptorConfig,string,TiXmlNode,DetectorDescriptorConfigContainer,istringstream,cerr,exit}, 
      identifierstyle=\color{black},
      stringstyle=\color{blue},      
      numbers=left,                 
      numbersep=5pt,                  
      numberstyle=\tiny\color{black}, 
      rulecolor=\color{black},        
      showspaces=false,               
      showstringspaces=false,        
      showtabs=false,                
      stepnumber=1,                   
      tabsize=5,                     
      title=\lstname,                 
    }

\begin{lstlisting} [language=cpp,caption={Controle cinemático com lei proporcional com feedforward.}\label{lst:pplusff}]
void doris_manipulator_controller::DorisManipulatorController::
	PplusFFMat()
{
    double tnow = GetTimeMicro();
    double t = (tnow - start)/1000000.0;
    Eigen::Vector4d xd, xdd, error, ubar, u;

    double* p = JuliaFunctionWrapper(t);
    xd = Eigen::Map<Eigen::Vector4d>(p);
    xdd =  Eigen::Map<Eigen::Vector4d>(p+4);

    Eigen::Matrix4d Ji = Ja_b.inverse();

    error = xd - x;
    ubar = kt*error + xdd;
    u = Ji*ubar;

    for (int i = 0; i < outputs.size(); i++)
        outputs[i] = u(i);
    SaturateOutput();
}
\end{lstlisting}

\newpage 
\begin{lstlisting} [language=cpp,caption={Controle por servovisão.}\label{lst:visioncontrol}]
void doris_manipulator_controller::DorisManipulatorController::
	VisionControl()
{
    if (trackingStatus == TRACKER_STATE_TRACK_MODEL)
    {
        Eigen::Vector4d x_delta, x_et, u, ubar;
        Eigen::Vector3d p_ct(x_target(0), x_target(1), x_target(2));

        Eigen::Matrix3d Rec;
        Rec << 0,  0, 1,
               0, -1, 0,
               1,  0, 0;

        Eigen::Vector3d p_et = Rec * p_ct;
        x_et  << p_et(1), p_et(2), p_et(3), x_target(3);
        x_delta = x_et - x_elt;

        ubar = kv * x_delta;
        Eigen::Matrix4d Ji = Ja_e.inverse();
        u = Ji*ubar;

        for (int i = 0; i < outputs.size(); i++)
            outputs[i] = u(i);
        SaturateOutput();
    }
    else
    {
        for (int i = 0; i < outputs.size(); i++)
            outputs[i] = 0;
    }
}
\end{lstlisting}

\newpage 
\begin{lstlisting} [language=cpp,caption={Controle de Força.}\label{lst:forcectrl}]
void doris_manipulator_controller::DorisManipulatorController::
	ForceControl()
{
    Eigen::Vector4d ubar, u;
    Eigen::Vector3d error, Fd, ubar_f;
    Fd = Eigen::Vector3d(inputs[0],inputs[1],inputs[2]);

    Eigen::Matrix3d Res, Sf;
    Res << 0,  0, 1,
           0, -1, 0,
           1,  0, 0;

    Sf <<  1, 0, 0,
           0, 0, 0,
           0, 0, 0;

    Eigen::Vector3d Fflt_e = Res*Fflt_s;
    error = Fflt_e - Fd;
    ubar_f = (pid->calculate(error))/KS;
    ubar << Sf * ubar_f, 0;
    Eigen::Matrix4d Ji = Ja_e.inverse();
    u = Ji*ubar;

    for (int i = 0; i < outputs.size(); i++)
        outputs[i] = u(i);
    SaturateOutput();
}
\end{lstlisting}