\chapter{Cinemática}

\section{Posição e orientação de um corpo rígido}

Um corpo rígido é descrito no espaço por sua posição e orientação (\textit{pose}) em relação a um sistema de coordenadas de referência. Escolhe-se um ponto do corpo e afixa-se um sistema de coordenadas. Denota-se como $\bar{E}$ um sistema de coordenadas ortonormal com $\vec{x}$, $\vec{y}$ e $\vec{z}$ como vetores unitários.

Sejam o sistema de coordenadas inercial $\bar{E}_a = [\vec{x}_a \; \vec{y}_a \; \vec{z}_a ]$ e o sistema de coordenadas do corpo $\bar{E}_b = [\vec{x}_b \; \vec{y}_b \; \vec{z}_b ]$.


As coordenadas de $\vec{x}_b$, $\vec{y}_b$ e $\vec{z}_b$ representadas no sistema de coordenadas $\bar{E}_a$ são

\begin{align}
\vec{x}_b &= \bar{E}_a x_{ab} \\
\vec{y}_b &= \bar{E}_a y_{ab} \\
\vec{z}_b &= \bar{E}_a z_{ab} ,
\end{align}
portanto,
\begin{equation}
\bar{E}_b = [\bar{E}_a x_{ab} \;\; \bar{E}_a y_{ab} \;\; \bar{E}_a z_{ab}] = \bar{E}_a [x_{ab} \;\;  y_{ab} \;\; z_{ab}] = \bar{E}_a R_{ab}
\end{equation}

A matriz $R_{ab}$ é chamada de matriz de rotação.
\begin{equation}
R_{ab} = \m{ x_{ab} & y_{ab} & z_{ab} }
\end{equation}
onde $x_{ab} \in \mathbb{R}^3$,  $y_{ab} \in \mathbb{R}^3$ e $z_{ab} \in \mathbb{R}^3$ são as componentes do sistema de coordenadas $\bar{E}_b$ no sistema de coordenadas $\bar{E}_a$, ou seja:


\begin{equation}
R_{ab} =  \m{ \vec{x}_a \cdot \\ \vec{y}_a \cdot  \\ \vec{z}_a \cdot  } \m{ \vec{x}_b & \vec{y}_b & \vec{z}_b } = 
\m{
	(\vec{x}_a \cdot \vec{x}_b) & (\vec{x}_a \cdot \vec{y}_b)& (\vec{x}_a \cdot \vec{z}_b) \\
	(\vec{y}_a \cdot \vec{x}_b)& (\vec{y}_a \cdot \vec{y}_b)& (\vec{y}_a \cdot \vec{z}_b) \\
	(\vec{z}_a \cdot \vec{x}_b) &(\vec{z}_a \cdot \vec{y}_b)& (\vec{z}_a \cdot \vec{z}_b)
}
\end{equation}



A posição da origem do sistema de coordenadas do corpo rígido $O_b$ pode ser expressa no sistema de coordenadas inercial como 
\begin{equation}
p_b = p_{bx} 
\end{equation}