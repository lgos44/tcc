\chapter{Conclusões e Trabalhos Futuros}

\section{Conclusões}
Com base nos resultados obtidos:

Quando se necessita de ROS nem sempre é uma solução adequada,

\section{Trabalhos futuros}
\begin{itemize}
\item Extender o uso do Julia Language, de modo a tornar o controle ainda mais dinâmico. Atualmente é utilizado somente na definição de uma trajetória a ser rastreada, no entanto é possível implementar algoritmos de controle inteiramente no Julia, o que permitiria ajustes e experimentos em tempo de execução.
\item Utilizar ambos os sensores da cãmera estereoscópica Minoru de modo a melhorar a estimação da posição e orientação de um alvo, assim como ampliar o campo de visão.
\item Modelo dinâmico do manipulador TETIS. 
\item Integrar o \textit{Sensable Phantom Omni}, um dispositivo háptico, em modo Master/Slave
\end{itemize}