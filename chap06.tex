\chapter{Conclusões e Trabalhos Futuros}

\section{Conclusões}

Neste projeto de graduação foi apresentada a modelagem cinemática, estratégias de controle baseadas na abordagem de controle cinemático, assim como a implementação de um software para controle de sistemas robóticos, utilizando como estudo de caso o manipulador TETIS, do sistema DORIS. 

A modelagem cinemática consiste em representar o manipulador como uma cadeia cinemática de corpos rígidos, de modo a obter relações geométricas que governam o sistema. A cinemática direta do manipulador TETIS foi obtida através da convenção Denavit–Hartenberg. Diferenciando as equações resultantes com respeito as variáveis das juntas foi possível obter a cinemática diferencial, relacionando a velocidade no espaço operacional com as variáveis de juntas, na forma do Jacobiano analítico. Em simulação os resultados se mostraram compatíveis com o esperado. 

A abordagem de controle cinemático assume que é possível considerar um sistema cinemático como entrada do sistema, tipicamente um valor de velocidade. Isso é possível quando existe uma malha de controle de baixo nível, que idealmente é capaz de impor uma velocidade especificada de referência. Esse é o caso do TETIS, que utiliza atuadores \textit{FHA Mini Servo} com drivers EPOS2 70/10, com uma malha de baixo nível de velocidade. Assim, utilizou-se como modelo simplificado do robô um integrador, supondo que os servos são capazes de reproduzir comandos de velocidade de forma razoavelmente precisa. No caso da junta 2, essa reprodução não é tão precisa devido necessidade de maior torque, por estar no início da cadeia.

O rastreamento de trajetória com controle proporcional com \textit{feedforward} apresentou erro em regime permanente pelos seguintes fatores: (i) ciclo de controle máximo que o computador embarcado suporta é de $10ms$; (ii) Junta 2 não é capaz de reproduzir com tanta precisão comandos de velocidade, por estar mais sujeita à torques maiores; (iii) Ciclo de controle está sujeito a atrasos devido ao software e hardware não respeitarem restrições de tempo-real. 

Os componentes de software para controle, interação e configuração de parâmetros foram desenvolvidos e integrados ao RobotGUI, que já era utilizado para controle dos demais sistemas do robô. A interface gráfica permite a visualização de dados de forma gráfica \textit{online} dos sinais, assim como a reconfiguração de parâmetros no computador embarcado com aquelas desejadas. A adição de novas funcionalidades e modos de controle é imediata, devido a arquitetura modular e ao paradigma orientado a objetos. O ROS segue os princípios de um microkernel, onde as funcionalidades são implementadas separadamente em módulos bem definidos que se comunicam, em contraste com um "monólito" que realiza todas as funções. Isso permite que nós possam ser substituídos ou modificados sem influenciar outros, desde que sigam o mesmo protocolo (tópicos/serviços). Por exemplo, caso seja de interesse substituir o módulo de visão computacional basta publicar/subscrever aos mesmos tópicos.

A implementação de software para controle com computação em tempo real é custosa em tempo de implementação e pouco expansível, necessitando tratar de algoritmos de \textit{scheduling} e latência de entrada/saída \citep{nilsson1998real}. A abordagem utilizada neste trabalho permite uma implementação mais desacoplada e menos interdependente. No entanto, existem algumas desvantagens do uso do ROS para um software completamente desacoplado. Existem \textit{overheads} de comunicação entre \textit{ROS Nodes}, que pode se tornar crítico no caso de separar o nó de controle do nó de atuação. Essa separação introduziria atrasos significativos, tornando-se evidente para o caso em que o objetivo de controle é rastrear uma trajetória que é função do tempo. Em casos como nó de visão computacional que realiza o processamento de imagem a separação é inevitável e não representa um problema. Conclui-se ser mais apropriado juntar os nós de atuação e controle, o que melhorou significativamente o erro de rastreamento. %, já que o tempo de processamento já é 

Com o uso do Julia Language foi possível alterar em tempo de execução a trajetória a ser rastreada escrevendo as equações em uma janela na interface, com uma sintaxe simples. O \textit{script} é compilado \textit{just-in-time} e pode ser executado sem perda de performance.
%Com base nos resultados obtidos:
%Quando se necessita de ROS nem sempre é uma solução adequada,

\section{Trabalhos futuros}
\begin{itemize}
\item Modelo dinâmico do manipulador TETIS.

\item Utilizar ambos os sensores da cãmera estereoscópica Minoru de modo a melhorar a estimação da posição e orientação de um alvo, assim como ampliar o campo de visão.

\item Extender o uso do Julia Language, de modo a tornar o controle ainda mais dinâmico. Atualmente é utilizado somente na definição de uma trajetória a ser rastreada, no entanto é possível implementar algoritmos de controle inteiramente no Julia, o que permitiria ajustes e experimentos em tempo de execução.

\item Estudar implementações para sistemas de controle em tempo real e como tratar atrasos de entrada/saída.

\item Implementar a estratégia de controle híbrido posição-força.

\item Integrar o \textit{Sensable Phantom Omni}, um dispositivo háptico, em modo Master/Slave.
\end{itemize}

%A cinemática direta expressa a posição e orientação do efetuador final em função das variáveis de junta, enquanto que a cinemática diferencial fornece a relação entre a velocidade das juntas e a velocidade do efetuador final.