%!TEX root = <main.tex>
\chapter{Revisão Bibliográfica}

Neste capitulo serão abordados os conceitos necessários para modelagem e controle de manipuladores robóticos.

\section{Cinemática Direta}
Um manipulador robótico é composto de uma série de corpos rígidos denominados \textit{elos} conectados através de \textit{juntas}. 
Juntas podem ser:
\begin{itemize} 
\item Revolução
\item Prismática
\end{itemize}
Essa estrutura é chamada de cadeia cinemática.
Um extremo da cadeia é fixado a base e o outro ao efetuador.
Nesse texto serão abordadas apenas cadeias cinemáticas abertas, ou seja, aquelas em que existe apenas uma sequêcia de elos conectando os dois extremos da cadeia.
Cada junta acrescenta um grau de liberdade (DOF), ao qual está associado a uma variável de junta. No caso de uma junta de revolução um ângulo e no caso de uma junta prismática um deslocamento.
O objetivo da cinemática direta é calcular a posição e orientação do efetuador em função das variáveis das juntas.

%É possível expressar a 


\subsection{Jacobiano Geométrico}
\subsection{Jacobiano Analítico}
Quando a posição e orientação do efetuador são dadas em função de um número mínimos de parametros no espaço operacional é possível computar o Jacobinano pela diferenciação das equações da cinemática direta em função das variáveis das juntas.
Para isso utiliza-se a técnica analítica.

Seja $\bm{p}_e$ a posição do sistema de coordenadas do efetuador representada no sistema de coordenadas da base. O vetor $\dot{\bm{p}}_e$ é portanto a velocidade de translação, ou linear.
\begin{equation}
\dot{\bm{p}}_e = \frac{\partial \bm{p}_e }{\partial \bm{q}} \dot{\bm{q}} = \bm{J}_P (\bm{q}) \dot{\bm{q}} 
\end{equation}

Para a velocidade angular, pode ser considerada uma representação mínima da orientação em função de três variáveis $\phi_e$. 
A derivada no tempo $\dot{\bm{\phi}}_e$ não é igual a velocidade angular, no entanto, conhecida a função $\bm{\phi}_e(\bm{q})$:

\begin{equation}
\dot{\bm{\phi}}_e = \frac{\partial \bm{\phi}_e}{\partial \bm{q}} \bm{\dot{q}} = \bm{J}_{\phi}(\bm{q})\bm{\dot{q}}
\end{equation}

Sob essas premissas a cinemática diferencial pode ser obtida como:
\begin{equation} \label{eq:jacoba}
\bm{\dot{x}}_e = \m{\bm{\dot{p}}_e \\ \bm{\dot{\phi}}_e} = \m{\bm{J}_P(\bm{q}) \\ \bm{J_\phi}(\bm{q})} \bm{\dot{q}} = \bm{J}_A (\bm{q}) \dot{\bm{q}}
\end{equation}

 
\section{Controle Cinemático}
Se $x$ é uma representação da posição e orientação e $x_d$ o valor desejado nessa representação, seja o erro no espaço operacional:
\begin{equation}
\bm{e} = \bm{x}_d - \bm{x}
\end{equation}

Derivando em relação ao tempo
\begin{equation}
\bm{\dot{e}} = \bm{\dot{x}}_d - \bm{\dot{x}}
\end{equation}
podemos escrever a partir da equação \ref{eq:jacoba}:
\begin{equation}
\bm{\dot{e}} = \bm{\dot{x}}_d - \bm{J}_a(\bm{q})\dot{\bm{q}}
\end{equation}
Sendo $\bm{x}_d(t)$ uma trajetória desejada, deseja-se que $\bm{x}$ atinja $\bm{x}_d(t)$ em $t \to \infty$ .
A entrada de controle para o sistema é um valor de $\bm{u} = \dot{\bm{q}}$, logo, assumindo que $\bm{J}_A(q)$ é quadrada e não singular, a escolha da lei de controle
\begin{equation}
\bm{u} = \bm{J}_A^{-1}(\bm{q})\bar{\bm{u}}
\end{equation}
leva ao sistema linear:
\begin{equation}
\dot{\bm{e}} = \dot{\bm{x}}_d - \bar{\bm{u}}
\end{equation}
Se for escolhido $\bar{\bm{u}}$:
\begin{equation}
\bar{\bm{u}} = \dot{\bm{x}}_d + \bm{K} (\bm{x}_d - \bm{x})
\end{equation}
obtem-se a seguinte dinâmica para o erro
\begin{equation}
\dot{\bm{e}} + \bm{K} \bm{e} = 0
\end{equation}

