%!TEX root = <main.tex>
%\chapter{Revisão Bibliográfica}
\chapter{Conceitos de Robótica}
Neste capitulo serão abordados os conceitos necessários para modelagem e controle de manipuladores robóticos, focando no que foi utilizado neste trabalho para implementação no manipulador 4-DOF chamado de Tetis.

\section{Cinemática Direta}
Um manipulador robótico é composto de uma série de corpos rígidos denominados \textit{elos} conectados através de \textit{juntas}. 
Juntas podem ser:
\begin{itemize} 
\item Revolução
\item Prismática
\end{itemize}

Essa estrutura é chamada de cadeia cinemática.
Um extremo da cadeia é fixado a base e o outro ao efetuador.
Nesse texto serão abordadas apenas cadeias cinemáticas abertas, ou seja, aquelas em que existe apenas uma sequêcia de elos conectando os dois extremos da cadeia.
Cada junta acrescenta um grau de liberdade (DOF), ao qual está associado a uma variável de junta. No caso de uma junta de revolução um ângulo e no caso de uma junta prismática um deslocamento.
O objetivo da cinemática direta é calcular a posição e orientação do efetuador em função das variáveis das juntas.

%É possível expressar a 

Uma cadeia cinemática aberta é constituida por $n+1$ elos numerados de $0$ a $n$, onde o Elo 0 é fixado a base por convenção. O método utilizado consiste em definir um sistema de coordenadas associado a cada elo e calcular a transformação homogênea entre elos consecutivos. Em seguida a transformação do n-ésimo sistema de coordenadas pode ser obtida de forma recursiva como
\begin{equation}\label{eq:cinedireta}
\bm{T}_{0n}(\bm{q}) = \bm{T}_{01}(q_1) \bm{T}_{12}(q_{2}) {\dots} \bm{T}_{n-1,n}(q_n)
\end{equation}
onde $\bm{T}_{i-1,i}(q_i)$ denota a transformação homogênea do sistema de coordenadas solidário ao elo $i-1$àquele solídário ao elo $i$.

Logo a transformação homogênea que da efetuador final com respeito a base é dada por
\begin{equation}
\bm{T}_{be}(\bm{q}) = \bm{T}_{b0} \bm{T}_{0n}(\bm{q}) \bm{T}_{ne} 
\end{equation}

\subsection{Convenção Denavit-Hartenberg}
Para calcular a cinemática direta para uma manipulador de cadeia cinemática aberta de acordo com a equação \eqref{eq:cinedireta} um método sistemático foi definido para obter a relação entre a posição e orientação de dois elos consecutivos. A convenção Denavit-Hartenberg especifica um conjunto de regras sobre como definir os sistemas de coordenadas de cada elo.

Seja o Eixo $i$ o eixo da junta que conecta o elo $i-1$, ao elo $i$, então:

\begin{itemize}
\item Escolher o eixo $z_i$ ao longo do eixo da junta $i+1$.
\item Colocar a origem $O_i$ na interseção do eixo $z_i$ com a normal comum entre os eixos $z_{i-1}$ e $z_i$
\item Escolher $x_i$ ao longo da normal comum aos eixos $z_{i-1}$ e $z_i$, com direção da junta $i$ para a junta $i+1$. 
\item O eixo $y_i = z_i \times x_i$ é escolhido de forma a completar o sistema de coordenadas.
\end{itemize}

Essa convenção resulta em uma definição não única do sistema de coordenadas nos seguintes casos:

\begin{itemize}
\item Para o sistema de coordenadas $0$, somente a direção do eixo $z_0$ é especificada, portanto a escolha de $O_0$ e $ x_0$ é arbitrária.
\item Para o sistema de coordenadas $n$, como não existe junta $n+1$, $z_n$ não está definido, mas $x_n$ deve ser normal ao eixo $z_{n-1}$. Tipicamente escolhe-se $z_n$ alinhado com $z_{n-1}$.
\item Quando dois eixos consecutivos são paralelos, a normal comum entre eles não é definida de forma única. Tipicamente escolhe-se $O_i$ na junta $i+1$
\item  Quando dois eixos consecutivos se interceptam, direção de $x_i$ é normal e o sentido é arbitrário. Escolhe-se $O_i$ na intesecção.
\item Quando a junta $i$ é prismática a direção de $z_{i-1}$ é arbitrária.
\end{itemize}

\subsection{Espaço das Juntas e Espaço Operacional}
Para que o efetuador final de um manipulador realize alguma tarefa é necessário atribuir uma posição e orientação desejada, que  pode ser função do tempo. Surge então o problema de representar a orientação e  
Para descrever a posição e orientação do efetuador 
\section{Cinemática Diferencial}
\subsection{Jacobiano Geométrico}

\subsection{Jacobiano Analítico}
Quando a posição e orientação do efetuador são dadas em função de um número mínimos de parametros no espaço operacional é possível computar o Jacobinano pela diferenciação das equações da cinemática direta em função das variáveis das juntas.
Para isso utiliza-se a técnica analítica.

Seja $\bm{p}_e$ a posição do sistema de coordenadas do efetuador representada no sistema de coordenadas da base. O vetor $\dot{\bm{p}}_e$ é portanto a velocidade de translação, ou linear.
\begin{equation}
\dot{\bm{p}}_e = \frac{\partial \bm{p}_e }{\partial \bm{q}} \dot{\bm{q}} = \bm{J}_P (\bm{q}) \dot{\bm{q}} 
\end{equation}

Para a velocidade angular, pode ser considerada uma representação mínima da orientação em função de três variáveis $\phi_e$. 
A derivada no tempo $\dot{\bm{\phi}}_e$ não é igual a velocidade angular, no entanto, conhecida a função $\bm{\phi}_e(\bm{q})$:

\begin{equation}
\dot{\bm{\phi}}_e = \frac{\partial \bm{\phi}_e}{\partial \bm{q}} \bm{\dot{q}} = \bm{J}_{\phi}(\bm{q})\bm{\dot{q}}
\end{equation}

Sob essas premissas a cinemática diferencial pode ser obtida como:
\begin{equation} \label{eq:jacoba}
\bm{\dot{x}}_e = \m{\bm{\dot{p}}_e \\ \bm{\dot{\phi}}_e} = \m{\bm{J}_P(\bm{q}) \\ \bm{J_\phi}(\bm{q})} \bm{\dot{q}} = \bm{J}_A (\bm{q}) \dot{\bm{q}}
\end{equation}

 
\section{Controle Cinemático}
\label{sec:controle_cinematico}
O estratégia de controle cinemático pode ser aplicada quando considera-se que a dinâmica do manipulador pode ser desprezada. Essa hipótese se sustenta quando as seguintes premissas são válidas:
\begin{itemize}
\item Elevados fatores de redução nas juntas
\item Baixas velocidades na realização das tarefas
\item Existe uma malha de controle de velocidade de alto desempenho em cada junta
\end{itemize}

A maioria dos manipuladores possui uma malha de controle de velocidade em nível de juntas como na figura \ref{fig:controlejuntas}. Logo, para um controle de alto ganho temos que:
\[ u \approx \dot{\theta}\]
\begin{figure}[h!]
\centering
\begin{tikzpicture}[auto, node distance=2cm,>=latex']
    % We start by placing the blocks
    \node [input, name=input] {};
    \node [sum, right of=input] (sum) {};
    \node [block, right of=sum] (K) {$K$};
    \node [block, right of=K] (PWM) {PWM};
    \node [block, right of=PWM] (Robo) {Robô};
    \node [block, right of=Robo] (JA) {$\bm{J}_A$};
    \node [block, right of=JA] (Integral) {$\int$};
    \node [tmp, below of=K] (tmp1){};
    \node [output, right of=Integral] (output) {};

    % Once the nodes are placed, connecting them is easy. 
    \draw [draw,->] (input) -- node {$u$} (sum);
    \draw [->] (sum) -- node {$e$} (K);
    \draw [->] (K) -- node {$v$} (PWM);
    \draw [->] (PWM) -- node [name=tau] {$\tau$} (Robo);
    \draw [->] (Robo) -- node [name=dtheta] {$\dot{\theta}$} (JA);
    \draw [->] (JA) -- node {$\dot{x}$} (Integral);
    \draw [->] (Integral) -- node [name=x] {$x$}(output);
    \draw [->] (dtheta) |- (tmp1)-| node[pos=0.99] {$-$} (sum);
\end{tikzpicture}
\caption{Diagrama de Blocos: Malha de Controle de Velocidade a nível de juntas.}
\label{fig:controlejuntas}
\end{figure}


Portanto é possível implementar o controle cinemático segundo o diagrama \ref{fig:controlecinematico} 	

\begin{figure}[h!]
\centering
\begin{tikzpicture}[auto, node distance=2cm,>=latex']
    % We start by placing the blocks
    \node  [input, name=input2] {};
    \node at (0,-1) [input, name=input] {};
    \node [sum, right of=input] (sum) {};
    \node [block, right of=sum] (K) {$\bm{K}$};
    \node [sum, right of=K, node distance=2cm] (sum2) {};
    \node [tmp, above of =sum2, node distance=1cm] (tmp1){};
    \node [block, right of=sum2] (JA) {$\bm{J}_A^{-1}$};
    \node [block, below of=JA] (k) {$\bm{k}(\cdot)$};
    \node [block, right of=JA] (Integral) {$\int$};
    \node [tmp, above of=JA, node distance=1cm] (tmp2){};
    \node [output, right of=Integral] (output) {};

    % Once the nodes are placed, connecting them is easy. 
    \draw [draw,->] (input) -- node {$\bm{x}_d$} (sum);
    %\draw [draw,->] (input2) -- node {$u$} (sum2);
    \draw [draw,->] (input2) -- node [pos=0.1] {$\bm{\dot{x}}_d$} (tmp1)-| node [pos=0.8,anchor=left,left] {$+$} (sum2);
    \draw [->] (sum) -- node {$\bm{e}$} (K);
    \draw [->] (K) -- node {}  node[pos=0.8] {$+$} (sum2);
    \draw [->] (sum2) -- node [name=tau]  {} (JA);
    \draw [->] (JA) -- node [name=dtheta] {$\dot{\bm{q}}$} (Integral);
    \draw [->] (Integral) -- node [name=x] {$\bm{q}$}(output);
    \draw [->] (x) |- (k);
    \draw [->] (k) -| node[pos=0.99] {$-$} node [near end] {$\bm{x}$} (sum);
    \draw [->] (x) |- (tmp2) -| (JA);
    %\draw [->] (output) |- (tmp1)-| node[pos=0.99] {$-$} (sum);
\end{tikzpicture}
\caption{Diagrama de Blocos: Controle Cinemático Proporcional com FeedForward}
\label{fig:controlecinematico}
\end{figure}


Se $x$ é uma representação da posição e orientação e $x_d$ o valor desejado nessa representação, seja o erro no espaço operacional:
\begin{equation}
\bm{e} = \bm{x}_d - \bm{x}
\end{equation}

Derivando em relação ao tempo
\begin{equation}
\bm{\dot{e}} = \bm{\dot{x}}_d - \bm{\dot{x}}
\end{equation}
podemos escrever a partir da equação \ref{eq:jacoba}:
\begin{equation}
\bm{\dot{e}} = \bm{\dot{x}}_d - \bm{J}_a(\bm{q})\dot{\bm{q}}
\end{equation}
Sendo $\bm{x}_d(t)$ uma trajetória desejada, deseja-se que $\bm{x}$ atinja $\bm{x}_d(t)$ em $t \to \infty$ .
A entrada de controle para o sistema é um valor de $\bm{u} = \dot{\bm{q}}$, logo, assumindo que $\bm{J}_A(q)$ é quadrada e não singular, a escolha da lei de controle
\begin{equation}
\bm{u} = \bm{J}_A^{-1}(\bm{q})\bar{\bm{u}}
\end{equation}
leva ao sistema linear:
\begin{equation}
\dot{\bm{e}} = \dot{\bm{x}}_d - \bar{\bm{u}}
\end{equation}
Se for escolhido $\bar{\bm{u}}$:
\begin{equation}
\bar{\bm{u}} = \dot{\bm{x}}_d + \bm{K} (\bm{x}_d - \bm{x})
\end{equation}
obtem-se a seguinte dinâmica para o erro
\begin{equation}
\dot{\bm{e}} + \bm{K} \bm{e} = 0
\end{equation}

\section{Servo Visão}
A tarefa proposta na Servo Visão é controlar a posição e orientação do efetuador do manipulador, em relação a um alvo, usando características visuais extraidas de uma imagem. A câmera pode ser carregada pelo manipulador (montada no efetuador) ou colocada em um ponto fixo, observando tanto o efetuador como o alvo.

\subsection{Servo Visão Baseada em Posição}
Em um sistema de servo visão baseado em posição a posição e orientação do alvo com respeito a câmera $\bm{T}_{CT}$ é estimada. O problema de estimação da posição e orientação é discutido no apêndice ?.
Especifica-se uma posição desejada relativa ao sistema de coordenadas do alvo  $\bm{T}_{C^*T}$ e deseja-se determinar o movimento necessário para mover a câmera para a posição desejada, que chamamos de $\bm{T}_\delta$.

\begin{equation}
 \bm{T}_{CT} =  \bm{T}_\Delta \bm{T}_{C^*T}
\end{equation}

\begin{equation}
 \bm{T}_\Delta  =   \bm{T}_{CT} \bm{T}_{C^*T}^{-1}
\end{equation}