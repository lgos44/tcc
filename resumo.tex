\begin{abstract}

Apresenta-se, neste projeto de graduação, a modelagem e controle cinemáticos de um manipulador de quatro graus de liberdade, com aplicação no manipulador TETIS, desenvolvido no projeto DORIS. Expõe-se as estratégias de controle desenvolvidas e implementadas através de abordagem cinemática, destacando-se controle proporcional com feedforward para rastreamento de trajetórias; controle por servovisão baseado em posição utilizando algoritmos de visão computacional e controle proporcional integral de força.
Detalha-se o desenvolvimento e aquitetura de um software para controle de manipuladores robóticos utilizando Robot Operating System e Qt como \textit{frameworks}. Por fim discute-se os resultados de simulação e experimentais obtidos.

\end{abstract}